\documentclass[conference]{IEEEtran}
\IEEEoverridecommandlockouts
% The preceding line is only needed to identify funding in the first footnote. If that is unneeded, please comment it out.
\usepackage{cite}
\usepackage{amsmath,amssymb,amsfonts}
\usepackage{graphicx}
\usepackage{textcomp}
\usepackage{xcolor}
\def\BibTeX{{\rm B\kern-.05em{\sc i\kern-.025em b}\kern-.08em
    T\kern-.1667em\lower.7ex\hbox{E}\kern-.125emX}}
    \title{
    \vspace{1cm}
    {\includegraphics[width=0.15\textwidth]{iith.jpg} \\ 7447 BCD - Seven segment Display Decoder Assignment} }
    \author{manne laxman\\ Roll No: FWC22297 \\mannelaxman135@gmail.com}
     \begin{document}
     \maketitle
      \section {ABSTRACT}
       This paper shows how to use the 7447 BCD-Seven segment Display Decoder to learn Boolean logic using arduino uno.

       \section{COMPONENTS}
       The required components list is given in Table: I. and seven segment display is shown in Fig.2 and IC 7447 diagram is shown in Fig.1.
       \vspace{0.3cm}
        \begin{table} [htbp]
	\centering
	\begin{tabular}{| c | c | c |} \hline
	Components & Value & Quantity \\\hline
	IC & 7447 & 1 \\ \hline
	seven segment display & & 1\\ \hline
	Arduino & UNO & 1 \\ \hline
	Jumper Wires &  & 10 \\ \hline
	Breadboard & & 1 \\
	\hline
	\end{tabular}
	\vspace{0.3cm}
	\caption{\label{tab:widgets}}
	\end{table}

	\begin{figure}[h]
	\centering
	\includegraphics[width=0.5\textwidth]{imm.jpg}
	\caption{\label{fig-1:Gates}}
	\end{figure}

	\begin{figure}[h]
	\centering
	\includegraphics[width=0.3\textwidth]{seg.jpg}
	\caption{\label{fig-2:Gates}}
	\end{figure}
	\section{PROCEDURE}


	\begin{enumerate}
	\item Make the connections of 7447 IC and seven segment display as per b
	elow Fig.3.                                                                     \begin{figure}[h]                                                       \centering
	        \includegraphics[width=0.3\textwidth]{image1.jpg}
		        \caption{\label{fig-3:Gates}}
			\end{figure}

			\item Make the connections of 7447 IC and arduino uno as per below Fig.4.
			\begin{figure}[h]
			\centering
			\includegraphics[width=0.3\textwidth] {image2.jpg}
			\caption{\label{fig-4:Gates}}
			\end{figure}

			\item {Truth Table for incrementing from $0$ to $9$ in seven segment dis
			play }
			        \vspace{0.4cm}

				\begin{table}[htbp]
				    \centering
				    \begin{tabular}
				    { | c | c | c | c | c | c | c | c | } \hline
				    $Z$ & $Y$ & $X$ & $W$ & $D$ & $C$ & $B$ & $A$\\\hline
				    0   & 0   & 0   & 0   & 0  & 0 & 0  & 1 \\
				    0   & 0   & 0   & 1   & 0  & 0 & 1  & 0 \\
				    0   & 0   & 1   & 0   & 0  & 0 & 1  & 1 \\
				    0   & 0   & 1   & 1   & 0  & 1 & 0  & 0 \\
				    0   & 1   & 0   & 0   & 0  & 1 & 0  & 1 \\
				    0   & 1   & 0   & 1   & 0  & 1 & 1  & 0 \\
				    0   & 1   & 1   & 0   & 0  & 1 & 1  & 1 \\                              0   & 1   & 1   & 1   & 1  & 0 & 0  & 0 \\
				    1   & 0   & 0   & 0   & 1  & 0 & 0  & 1 \\
				    1   & 0   & 0   & 1   & 0  & 0 & 0  & 0 \\ \hline
				    \end{tabular}
				    \vspace{0.15cm}
				    \caption{\label{tab:widgets}}
				    \end{table}                                                            \item Execute the arduino code without any errors.
				    \item After upload the code into hardware setup using arduino IDE platfo
				    rm with hex file.
				     \end{enumerate}     
				     \section{RESULTS}
				      \begin{enumerate}
				               \item Download the code given in the link below and execute them to see the output as shown in Fig.5.
					                \item https://github.com/rajib05ra/FWC-Assignments/tree/main/Assignment
							 \end{enumerate}
							         \begin{figure}[h]
								         \centering
									         \includegraphics[width=0.35\textwidth]{bcdr.jpg}                                                     \caption{\label{fig:Gates}}
										 \end{figure}
										 \section{CONCLUSION}
										 Hence implementation of 7447 IC, Seven segment dispaly using arduino UNO is done.
										 \end{document}
